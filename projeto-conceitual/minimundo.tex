\documentclass[12pt, a4paper]{article}
%\usepackage[latin1]{inputenc}
\usepackage[utf8]{inputenc}
\usepackage[brazil]{babel}
\usepackage{indentfirst}
\usepackage{setspace}
\begin{document}

\begin{titlepage}
\begin{center}
{\large Universidade Federal de Pernambuco}\\[0.2cm]
{\large Centro de Informática}\\[0.2cm]
{\large Gerenciamento de Dados e Informação}\\[3.7cm]
{\bf \huge MINIMUNDO}\\[5.cm]
\end{center}
{\large {\bf Equipe:}\\[0.2cm]
Enzo Gurgel Bissoli (egb2)\\[0.2cm]
Pedro Grisi Oliveira de Queiroz (pgoq)\\[0.2cm]
Vinícius Lima Ventura (vlv2)\\[0.2cm]}
{\large {\bf Professor:} Robson do Nascimento Fidalgo}\\[2cm]
\begin{center}
{\large Recife}\\[0.2cm]
{\large 2021}\\[0.2cm]
\end{center}
\end{titlepage}
\onehalfspacing
\section{Descrição do Minimundo}

Um \textbf{funcionário} tem \underline{CPF}, nome, salário e pode ser empregado terceirizado ou não.  Um \textbf{funcionário} pode ser gerente de outros \textbf{funcionários}, os quais são gerenciados por um único funcionário.

Um \textbf{funcionário} pode ser funcionário \textbf{da cozinha} ou \textbf{do atendimento}. Ele trabalha para um único \textbf{estabelecimento}. Tem varios \textbf{funcionários} que trabalham no \textbf{estabelecimento}, o qual tem um \underline{id} de identificação, um nome, a quantidade de estabelecimentos daquele tipo e um endereço. O \textbf{estabelecimento} precisa ter um único \textbf{estoque} para armazenar os produtos, o \textbf{estoque} tem um nome e uma quantidade armazenada no estoque.

Os \textbf{produtos} são entregues aos \textbf{estabelecimento} por \textbf{caminhões}, cada \textbf{caminhão} tem uma \underline{placa} e a entrega precisa ter um \underline{código} associado a ela e uma data na qual deve ocorrer. Os \textbf{produtos} que são entregues tem um \underline{id} identificador, um nome e uma quantidade.

O \textbf{estabelecimento} tem um único \textbf{dono} que pode coordenar vários estabelecimentos. O dono tem duas formas de gerir seu negócio, através de uma \textbf{matriz} que tem \underline{CNPJ} ou \textbf{franquia} que tem o \underline{CPF} do dono e um contato (formado por email e telefone).

Os \textbf{produtos} são usados para preparar os \textbf{pratos} dos \textbf{clientes}, os \textbf{pratos} são preparados pelos \textbf{funcionários da cozinha}, cada \textbf{prato} tem um \underline{id} identificador para ser vendido pelos \textbf{funcionários do atendimento}, nessas vendas um \textbf{desconto} pode ser aplicado, existem diferentes \undeline{tipos} de desconto. Quando o \textbf{prato} é vendido para o \textbf{cliente}, é necessário ter o \underline{CPF} do cliente.


\end{document}